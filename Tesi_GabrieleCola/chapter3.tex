\chapter{Case studies of Imbalanceness}

\section{Description of Dataset}
The Dataset adopted for showing the Imbalanceness were : \noindent \\
1.Customers Churn \noindent \\
2.Spam \noindent \\

This Datasets were commonly used in the Imbalance Classification problem because they often show a minority class, which is almost always the 'yes' class~\cite{zhu2019iric}.  



\section{Churn Analysis}
Churn dataset contains 19 variables, which explain the features of customers.
 All data-level methods are applied in this dataset, namely all pre-processing methods that after they will be combined with SVM classifier in order to predict if a customer will leave the phone company or not.\noindent \\

\begin{center}
\includegraphics[width=1.1\textwidth]{Tesi_GabrieleCola/img/churn.pdf}
\end{center}

The highest accuracy is obtained  when the classifier is combined with OSS method, while Adasyn resulted as the worst method.
Furthermore, Under-sampling methods work better than Oversampling and Hybrid methods. \noindent \\

\section{Spam Detection}
Spam Dataset contains 58 variables,which explain the features of email. \noindent \\
The first 48 variables contain the frequency of the variable name in the e-mail. \noindent \\
The variables 49-54 indicate the frequency of the characters.\noindent \\
The variables 55-57 contain the average, longest and total run-length of capital letters.\noindent \\
 the variable 58 indicates the type of the mail and is either "nonspam" or "spam".\noindent \\

In this dataset, differently from Churn dataset, are applied all cost-sensitive algorithms, so there isn't a pre-processing of data but only modified algorithm, that will be used to predict if an email
will go to spam box or not.\noindent \\

\begin{center}
\includegraphics[width=1.1\textwidth]{Tesi_GabrieleCola/img/spam.pdf}
\end{center}

The classifier that has the highest accuracy is Smote-Boosting, while Cost-Sensitive Decision Tree resulted as the worst method.
Furthermore, in this case we can't affirm that a kind of method works better than others.


    
    



